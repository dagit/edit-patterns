\section{Introduction}
%% 1. Describe the problem

Version control systems (VCS's) track the evolution of software over time in the form
of a sequence of changes to the plain text representation of the code.
We would like to be able to characterize the changes to files in a software
project according to the type of change that they represent.  The ability to
map these changes to the syntax of the language, instead of its raw text
representation, will allow them to be understood in terms of the language
constructs themselves.  Doing so will allow us to identify patterns of
changes at the abstract syntax level to support insight at a deeper level
than achievable by examining raw text differences.  For example, in
an operating system kernel, a set of drivers may need to be updated to use a
new mutual exclusion mechanism. In this example, blocks of code that access
specific resources need to be updated to take a semaphore before proceeding.

Finding common patterns for the changes to a source file gives us the ability
to understand, at a higher level, what sorts of revisions are happening. If we
already know to look for changes to semaphore handling we may think to use a
textual search tool, such as {\tt grep}, to search the source code for
functions related to semaphores. Instead, suppose we have the challenge that we
are not familiar with the source code and we would like to understand at a
structural level what types of changes were needed to add support for the new
semaphores.

Our contributions described in this work are:

\begin{itemize}

\item We show that structural differencing algorithms that operate on the
abstract syntax tree or parse tree of a language can be used to map text
differences stored in a VCS to a form where syntactic changes can be reasoned
about.

\item We show that the anti-unification algorithm of BLAH can be used to map
similar changes to a generalized change pattern.

\item We show that clustering by a tree similarity metric allows differences
to be grouped such that the families of changes to generalize via anti-
unification can be inferred from the differences using a basic similarlity
threshold.

\end{itemize}

In this paper, we briefly describe the building blocks of our work and show
preliminary results of this methodology as applied to version control
repositories for open source projects available online.  The projects studied
in this paper are ANTLR and Clojure, both written in Java.

\subsection{Related work}

The use of version control repositories as a source of data to track changes
over time is not new, but our approach to the problem is novel. 
These may be very
related~\cite{weissgerber06identify, kim07automatic, neamtiu05understand} - I
need to quickly re-scan these to see the overlap.  