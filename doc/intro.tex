\section{Introduction}
%% 1. Describe the problem

Version control systems (VCS's) track the evolution of software over time in
the form of a sequence of changes to the plain text representation of the
code. We would like to be able to characterize the changes to files in a
software project according to the type of change that they represent.  The
ability to map these changes to the syntax of the language, instead of its raw
text representation, will allow them to be understood in terms of the language
constructs themselves.  Doing so will allow us to identify patterns of changes
at the abstract syntax level, separate from syntax neutral changes to the text
such as layout variations.  As a result, the interpretation of changes is
made unambiguous given the definition of the abstract syntax of the language.

Finding common patterns for the changes to a source file gives us the ability
to understand, at a higher level, what sorts of revisions are happening.
Detecting simple changes, such as semaphore handling changes in system-level
software, we may think to use a textual search tool, such as {\tt grep}, to
search the source code for functions related to semaphores. Such tools are
unable to easily identify more complex patterns though that have no single
textual representation, such as instances of semaphore handling calls being made
within conditionals where the format of the conditional can vary.  Structure
aware searching would be necessary in this case, as treating the program as
raw text ignores important syntactic structure.

In an even more complicated situation, a programmer may be faced with a
code base that they are unfamiliar with.  In this case, the programmer may not
know a-priori what kinds of structures are important to look for related to
a certain kind of change.  Here, we would like to use the differences that are
recorded in the VCS during the period of time when the change of interest was
being performed to discover the structural patterns that represent the high
level structure of the changes.  In this way, our goal is to not provide simply
a sophisticated search tool, but to provide a method for identifying patterns 
of code changes over a period of time.

Our contributions towards this goal presented in this paper are:

\begin{itemize}

\item We show that structural differencing algorithms that operate on the
abstract syntax tree (AST) of a language can be used to map text
differences stored in a VCS to a form where the syntactic meaning of changes
can be reasoned about.

\item We show that the antiunification algorithm that seeks the ``least general
generalization'' of a set of trees can be used to map changes considered to be
sufficiently similar to a meaningful generalized change pattern.

\item We show that a thresholded tree similarity metric derived from
a tree edit distance score provides a useful grouping mechanism to define
the notion of ``sufficiently similar''.

\end{itemize}

In this paper, we briefly describe the building blocks of our work and show
preliminary results of this methodology as applied to version control
repositories for open source projects available online.  The projects studied
in this paper are ANTLR\footnote{\url{https://github.com/antlr/antlr4}} and
Clojure\footnote{\url{https://github.com/clojure/clojure}}, both written in
Java.

\subsection{Motivation}
\label{sec:motivation}

We would like to be able to take existing software projects and use the history
stored in the VCS to answer questions which may be important to software
developers, software project managers, language designers, and static analysis
tools.

% For example the manager of a large software project may lead an effort to port
% an existing application to multicore.  After a successful port an interesting
% question to ask might be, what types of changes were needed to add mutual
% exclusion? Another question might be, do seemingly simple changes require large
% changes in the source code and why?  Answers to these questions could help
% inform future software development on the code base, such as making decisions
% about when is the right time to undertake a large restructuring.

Language designers may want to know whether specific syntactic constructs would
make the language more productive for users. Taking an example from Java, we
might consider the addition of the for-each loop construct. This feature could
be partially justified by doing an analysis of existing source code to
determine that most for-loops iterate over an entire collection. To strengthen
this argument, it would be insightful to know what is the impact of maintaining
the code without for-each. For example, if refactoring the code commonly leads
to editing the bounds to match the collection used, then the argument in favor
of adding for-each is strengthened, as now it helps to prevent a class of bugs
where programmers forget to update the bounds. We demonstrate the detection of
loop patterns within ANTLR in Section~\ref{sec:groups}.

Software developers joining a new project or team are expected to learn the
source code that they will be working with. We would like to provide these
programmers with tools that aid them in this task by allowing them to see what
types of changes other team members have made in the past. Software developers
may also want to compare the changes that happen in response to related bugs,
hoping to find opportunities to improve software quality, either by searching
for buggy patterns in the source code or making a tool to detect the pattern in
new code. We demonstrate the detection of generic patterns within the
Clojure compiler in Section~\ref{sec:clojure}.

We believe there are many uses for this approach beyond the ones demonstrated
in this paper. Consider a problem that has been faced by many projects in the
last decade---the challenge of migrating to utilize multicore processors.  A
manager who is leading a large software project may want to answer important
questions to help inform future development: what sorts of constructs were
removed or added?  This can reveal patterns of code that were thread unsafe in
the pre-multicore code that developers (especially those not participating in
the multicore port) should be made aware of in the future.  It can also reveal
repeated patterns that were added, indicating potential refactorings that may
be desirable to apply in order to reduce the proliferation of code clones
within the project.


\subsection{Related work}

The use of version control repositories as a source of data to study changes
to code over time is not new, but our approach to the problem is novel.
Neamtiu~\cite{neamtiu05understand} uses a similar approach of analyzing the
abstract syntax tree of code in successive program versions, but focuses on
detecting change occurrences only instead of going a step further and
attempting to identify any common patterns of change that can be found.  
Other groups have focused on identifying patterns based on common
refactorings that can be identified in the code~\cite{weissgerber06identify},
and seek to infer simple abstract rules that encapsulate the changes
that they detect~\cite{kim07automatic}.  For example, one such rule could
indicate that for all calls that match a certain pattern, an additional
argument should be added to their argument list.

This goal of generating abstract rules is similar to our goal of inferring
generic patterns in terms via
antiunification~\cite{reynolds70antiunification,plotkin70antiunification}.
What differs with our approach is that we presuppose no knowledge of the
underlying language beyond the structure provided by the language parser and
its mapping to an annotated term (or, aterm)~\cite{brand00aterm} format.  As
such, it is challenging to build rules that give an interpretation to the
program abstract syntax, such as ``append an argument to the function call'',
since we do not provide a mapping from the concept of ``function call'' to a
pattern of AST nodes.  By instead emitting templates in terms of the language
AST in aterm form, we are able to keep the tool as language-neutral as
possible.
