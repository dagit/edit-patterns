\section{Experimental results}

We tested the methodology outlined in section~\ref{sec:method} on the
publicly available git repositories for two popular open source
projects, the ANTLR parser generator and the Clojure language implementation.
Both are implemented in Java, and one (ANTLR) is composed of a mixture of
hand-written and automatically generated code.  

\subsection{Threshold sensitivity}
\label{sec:threshold}

The first experiment that we performed was to investigate the effect of
similarity threshold to the number of groups identified, as well as the degree
of generality present in the tree that results from all members of each group
being antiunified together. Our prediction was that at the lowest threshold
(0.0), when all trees are considered to be similar, their antiunification
will yield the most general pattern.  This is what was observed, in which the
antiunification result is a tree composed of a single metavariable node.
Similarly, at the highest threshold (1.0), the only groupings that will be
present will be single tree sets, or sets containing identical trees for
instances of identical changes that occurred in different places.  This is
precisely what we observed, with the antiunified trees containing no meta-
variables since antiunification of a set of identical elements is the element
itself.  We show the number of groups (broken down by type: code addition,
code deletion, or code mutation) as a function of threshold in
Figure~\ref{fig:thresholdplot}.

As we can see, as the threshold increases, we see more groupings of changes
due to changes that were considered similar under a lower threshold being
considered dissimilar under the more restrictive threshold.  For example, at
$\tau=0.15$, a single pattern for for-loops is identified:

\begin{java}
for ($\metavar$=$\metavar$;$\metavar$<$\metavar$;$\metavar$) {
    $\metavar$
}
\end{java}

As the threshold is increased to $\tau=0.25$, in addition to generic for-loops, a
cohort of changes are identified to a more specific instance of the for-loop where
the loop counter is initialized to zero:

\begin{java}
for ($\metavar$=0;$\metavar$<$\metavar$;$\metavar$) {
    $\metavar$
}
\end{java}

Increasing to $\tau=0.35$, the pattern for the conditional becomes more specific
and we see what appears to be a template for using the field of an object
(e.g., {\tt args.length}) as the loop termination criterion:

\begin{java}
for ($\metavar$=0;$\metavar$<$\metavar$.$\metavar$;$\metavar$) {
    $\metavar$
}
\end{java}

Similar templates emerge for code patterns such as method invocations, printing
the concatenation of two strings, and other common activities.  

\jason{Talk about System.\metavar.\metavar("\metavar" + \metavar); pattern and how that evolves into
System.out.println("\metavar" + \metavar);?}

\subsection{Group sizes}
\label{sec:groups}

\jason{Are we talking about AU groups or the more broad types of changes groups?
I wrote this to be the latter, but they're both interesting and related.}

We also considered how the sizes of groups changed as we varied the threshold
of similarity ($\tau$). Figure~\ref{fig:clojure-number-of-modifications} shows
the size of the groups for the Clojure history and
Figure~\ref{fig:antlr-number-of-modifications} shows the size of the groups for
the ANTLR history. In both cases, we only consider a small portion of the full
history of the VCS.

\begin{figure}
\begin{center}
\includegraphics[width=0.44\textwidth]{figures/clojure-number-of-modifications.pdf}
\caption{Number of additions, deletions, and modifications by threshold for the Clojure source}
\label{fig:clojure-number-of-modifications}
\end{center}
\end{figure}

\begin{figure}
\begin{center}
\includegraphics[width=0.44\textwidth]{figures/antlr-number-of-modifications.pdf}
\caption{Number of additions, deletions, and modifications by threshold for the ANTLR source}
\label{fig:antlr-number-of-modifications}
\end{center}
\end{figure}

The graphs show us that as the threshold increases the number of distinct trees
increases. At the maximum threshold of 1, the total number of changes is less
than the number of trees we started with, because some changes end up being
identical.

\subsection{Pattern identification}

\subsubsection{Clojure}
\label{sec:clojure}

Using a portion of the Clojure history, we computed the number of similar trees
at each threshold. See figure~\ref{fig:clojure-number-of-modifications} to see
the number of changes by time as a function of the threshold. We looked at
thresholds from 0 to 1 with an increment size of 0.01.

Looking at just the number of deletions, we examined the point where the number
of deletions goes from 4 to 5 as the threshold changes from 0.35 to 0.36.

The following code, presented in a standard diff format, shows a loop and the
lines that were removed. This example comes from a file named {\tt
PersistentArrayMap.java}.

\begin{java}
 public Object kvreduce(IFn f, Object init){
     for(int i=0;i < array.length;i+=2){
         init = f.invoke(init, array[i], array[i+1]);
-           if(RT.isReduced(init))
-                   return ((IDeref)init).deref();
         }
     return init;
 }
\end{java}

Given the low threshold, this deletion was considered to be similar to the
following deletions. This example comes from {\tt PersistentHashMap.java}.
Note: Our parser ignores whitespace and gives the same AST for \verb|if (exp) { stmt; }|
and \verb|if (exp) stmt;|. In the diff below, we have prefixed the
lines that were considered to actually be different by our tool with ``>''
characters.

\begin{java}
 public Object kvreduce(IFn f, Object init){
-    for(INode node : array){
-        if(node != null){
+    for(INode node : array)
+        {
+        if(node != null)
             init = node.kvreduce(f,init);
>-                if(RT.isReduced(init))
>-                        return ((IDeref)init).deref();
-               }
-           }
+        }
     return init;
 }
\end{java}

In both cases, our tool identified for-loops where the same lines are removed.
In fact, the code for both of these is very similar perhaps owing to Java's
HashMap and ArrayMap classes being very similar in terms of interface.
Furthermore, it did this at the statement level, eg., we did not need to
consider the similarities of the file names or the method names.

\subsubsection{ANTLR}

One of the challenges with finding interesting patterns in the ANTLR source is
that some of the files are autogenerated. We would prefer to exclude
autogenerated files from the analysis.
