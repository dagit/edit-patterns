\documentclass{acm_proc_article-sp}

\begin{document}

\title{Looking for historical patterns }
\subtitle{[Extended Abstract]
\titlenote{A full version of this paper is available as
\textit{Author's Guide to Preparing ACM SIG Proceedings Using
\LaTeX$2_\epsilon$\ and BibTeX} at
\texttt{www.acm.org/eaddress.htm}}}
%
% You need the command \numberofauthors to handle the 'placement
% and alignment' of the authors beneath the title.
%
% For aesthetic reasons, we recommend 'three authors at a time'
% i.e. three 'name/affiliation blocks' be placed beneath the title.
%
% NOTE: You are NOT restricted in how many 'rows' of
% "name/affiliations" may appear. We just ask that you restrict
% the number of 'columns' to three.
%
% Because of the available 'opening page real-estate'
% we ask you to refrain from putting more than six authors
% (two rows with three columns) beneath the article title.
% More than six makes the first-page appear very cluttered indeed.
%
% Use the \alignauthor commands to handle the names
% and affiliations for an 'aesthetic maximum' of six authors.
% Add names, affiliations, addresses for
% the seventh etc. author(s) as the argument for the
% \additionalauthors command.
% These 'additional authors' will be output/set for you
% without further effort on your part as the last section in
% the body of your article BEFORE References or any Appendices.

\numberofauthors{2} %  in this sample file, there are a *total*
% of EIGHT authors. SIX appear on the 'first-page' (for formatting
% reasons) and the remaining two appear in the \additionalauthors section.
%
\author{
% You can go ahead and credit any number of authors here,
% e.g. one 'row of three' or two rows (consisting of one row of three
% and a second row of one, two or three).
%
% The command \alignauthor (no curly braces needed) should
% precede each author name, affiliation/snail-mail address and
% e-mail address. Additionally, tag each line of
% affiliation/address with \affaddr, and tag the
% e-mail address with \email.
%
% 1st. author
\alignauthor
Jason dagit\\
       \affaddr{Galois, Inc}\\
       \affaddr{Portland, OR}\\
       \email{dagit@galois.com}
% 2nd. author
\alignauthor
Matthew Sottile\\
       \affaddr{Galois, Inc}\\
       \affaddr{Portland, OR}\\
       \email{matt@galois.com}
}
\date{30 June 2013}
% Just remember to make sure that the TOTAL number of authors
% is the number that will appear on the first page PLUS the
% number that will appear in the \additionalauthors section.

\maketitle
\begin{abstract}
VCS History. Structural tree diff. Antiunification. Patterns.
\end{abstract}

% TODO: edit categories
% A category with the (minimum) three required fields
\category{H.4}{Information Systems Applications}{Miscellaneous}
%A category including the fourth, optional field follows...
\category{D.2.8}{Software Engineering}{Metrics}[complexity measures, performance measures]

\terms{Theory} % TODO: changeme

%TODO: do I need this?
\keywords{ACM proceedings, \LaTeX, text tagging} % NOT required for Proceedings

\section{Introduction}
Intro here.


\subsection{Citations}

\subsection{Tables}

\subsection{Figures}

\section{Conclusions}
\section{Acknowledgments}
This section is optional; it is a location for you
to acknowledge grants, funding, editing assistance and
what have you.  In the present case, for example, the
authors would like to thank Gerald Murray of ACM for
his help in codifying this \textit{Author's Guide}
and the \textbf{.cls} and \textbf{.tex} files that it describes.
%\end{document}  % This is where a 'short' article might terminate

%ACKNOWLEDGMENTS are optional

%
% The following two commands are all you need in the
% initial runs of your .tex file to
% produce the bibliography for the citations in your paper.
\bibliographystyle{abbrv}
\bibliography{sigproc}  % sigproc.bib is the name of the Bibliography in this case
% You must have a proper ".bib" file
%  and remember to run:
% latex bibtex latex latex
% to resolve all references
%
% ACM needs 'a single self-contained file'!
%
%% %APPENDICES are optional
%% %\balancecolumns
%% \appendix
%% %Appendix A
%% \section{Headings in Appendices}
%% The rules about hierarchical headings discussed above for
%% the body of the article are different in the appendices.
%% In the \textbf{appendix} environment, the command
%% \textbf{section} is used to
%% indicate the start of each Appendix, with alphabetic order
%% designation (i.e. the first is A, the second B, etc.) and
%% a title (if you include one).  So, if you need
%% hierarchical structure
%% \textit{within} an Appendix, start with \textbf{subsection} as the
%% highest level. Here is an outline of the body of this
%% document in Appendix-appropriate form:
%% \subsection{Introduction}
%% \subsection{The Body of the Paper}
%% \subsubsection{Type Changes and  Special Characters}
%% \subsubsection{Math Equations}
%% \paragraph{Inline (In-text) Equations}
%% \paragraph{Display Equations}
%% \subsubsection{Citations}
%% \subsubsection{Tables}
%% \subsubsection{Figures}
%% \subsubsection{Theorem-like Constructs}
%% \subsubsection*{A Caveat for the \TeX\ Expert}
%% \subsection{Conclusions}
%% \subsection{Acknowledgments}
%% \subsection{Additional Authors}
%% This section is inserted by \LaTeX; you do not insert it.
%% You just add the names and information in the
%% \texttt{{\char'134}additionalauthors} command at the start
%% of the document.
%% \subsection{References}
%% Generated by bibtex from your ~.bib file.  Run latex,
%% then bibtex, then latex twice (to resolve references)
%% to create the ~.bbl file.  Insert that ~.bbl file into
%% the .tex source file and comment out
%% the command \texttt{{\char'134}thebibliography}.
%% % This next section command marks the start of
%% % Appendix B, and does not continue the present hierarchy
%% \section{More Help for the Hardy}
%% The acm\_proc\_article-sp document class file itself is chock-full of succinct
%% and helpful comments.  If you consider yourself a moderately
%% experienced to expert user of \LaTeX, you may find reading
%% it useful but please remember not to change it.
%% \balancecolumns
%% % That's all folks!
\end{document}
