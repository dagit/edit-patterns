\documentclass{acm_proc_article-sp}

\newcommand{\claim}[1]{#1}

\begin{document}

\title{Looking for historical development patterns via version control}
\subtitle{[Extended Abstract]
\titlenote{A full version of this paper is available as
\textit{Author's Guide to Preparing ACM SIG Proceedings Using
\LaTeX$2_\epsilon$\ and BibTeX} at
\texttt{www.acm.org/eaddress.htm}}}
%
% You need the command \numberofauthors to handle the 'placement
% and alignment' of the authors beneath the title.
%
% For aesthetic reasons, we recommend 'three authors at a time'
% i.e. three 'name/affiliation blocks' be placed beneath the title.
%
% NOTE: You are NOT restricted in how many 'rows' of
% "name/affiliations" may appear. We just ask that you restrict
% the number of 'columns' to three.
%
% Because of the available 'opening page real-estate'
% we ask you to refrain from putting more than six authors
% (two rows with three columns) beneath the article title.
% More than six makes the first-page appear very cluttered indeed.
%
% Use the \alignauthor commands to handle the names
% and affiliations for an 'aesthetic maximum' of six authors.
% Add names, affiliations, addresses for
% the seventh etc. author(s) as the argument for the
% \additionalauthors command.
% These 'additional authors' will be output/set for you
% without further effort on your part as the last section in
% the body of your article BEFORE References or any Appendices.

\numberofauthors{2} %  in this sample file, there are a *total*
% of EIGHT authors. SIX appear on the 'first-page' (for formatting
% reasons) and the remaining two appear in the \additionalauthors section.
%
\author{
% You can go ahead and credit any number of authors here,
% e.g. one 'row of three' or two rows (consisting of one row of three
% and a second row of one, two or three).
%
% The command \alignauthor (no curly braces needed) should
% precede each author name, affiliation/snail-mail address and
% e-mail address. Additionally, tag each line of
% affiliation/address with \affaddr, and tag the
% e-mail address with \email.
%
% 1st. author
\alignauthor
Jason Dagit\\
       \affaddr{Galois, Inc}\\
       \affaddr{Portland, OR}\\
       \email{dagit@galois.com}
% 2nd. author
\alignauthor
Matthew Sottile\\
       \affaddr{Galois, Inc}\\
       \affaddr{Portland, OR}\\
       \email{mjsottile@gmail.com}
}
\date{30 June 2013}
% Just remember to make sure that the TOTAL number of authors
% is the number that will appear on the first page PLUS the
% number that will appear in the \additionalauthors section.

\maketitle
\begin{abstract}

Traditional algorithms for detecting differences in source code focus on
differences between lines. We propose to use the structure of the program
source to identify structural changes over the history of a software project.
We use a tree differencing algorithm on revisions stored in a Version Control
System (VCS) to identify structral changes. Once we have extracted the
structural differences we apply similarity metrics to find related changes. By
applying antiunification to similar changes we are able to generalize from
concrete changes to patterns of structural change. Furthermore, studying
patterns of change at the structural level, instead of line-by-line, allows us
to gain insight into the evolution of software.

\end{abstract}

% TODO: edit categories
% A category with the (minimum) three required fields
\category{H.4}{Information Systems Applications}{Miscellaneous}
%A category including the fourth, optional field follows...
\category{D.2.8}{Software Engineering}{Metrics}[complexity measures, performance measures]

\terms{Theory} % TODO: changeme

%TODO: do I need this?
\keywords{ACM proceedings, \LaTeX, text tagging} % NOT required for Proceedings

\section{Introduction}
%% 1. Describe the problem

Version control systems (VCS's) track the evolution of software over time in the form
of a sequence of changes to the plain text representation of the code.
We would like to be able to characterize the changes to files in a software
project according to the type of change that they represent.  The ability to
map these changes to the syntax of the language, instead of its raw text
representation, will allow them to be understood in terms of the language
constructs themselves.  Doing so will allow us to identify patterns of
changes at the abstract syntax level to support insight at a deeper level
than achievable by examining raw text differences.  For example, in
an operating system kernel, a set of drivers may need to be updated to use a
new mutual exclusion mechanism. In this example, blocks of code that access
specific resources need to be updated to take a semaphore before proceeding.

Finding common patterns for the changes to a source file gives us the ability
to understand, at a higher level, what sorts of revisions are happening. If we
already know to look for changes to semaphore handling we may think to use a
textual search tool, such as {\tt grep}, to search the source code for
functions related to semaphores. Instead, suppose we have the challenge that we
are not familiar with the source code and we would like to understand at a
structural level what types of changes were needed to add support for the new
semaphores.

Furthermore, 

\claim{Finding edit patterns in source code is a hard problem.}

%% 2. State our contributions

Our contributions described in this work are:

\begin{itemize}

\item We show that structural differencing algorithms that operate on the abstract
syntax tree or parse tree of a language can be used to map text differences stored in
a VCS to a form where syntactic changes can be reasoned about.

\item We show that the anti-unification algorithm of BLAH can be used to map
similar changes to a generalized change pattern.

\item We show that clustering by a tree similarity metric allows differences to be grouped
such that the families of changes to generalize via anti-unification can be inferred from
the differences using a basic similarlity threshold.

\end{itemize}

In this paper, we briefly describe the building blocks of our work and show preliminary results
of this methodology as applied to version control repositories for open source projects available
online.  The projects studied in this paper are X, Y, and Z.  In lieu of a stand-alone discussion of
related work, we will present the work that we have based our research on inline with the associated
topics.

\section{Methodology}

We propose the following workflow for studying software evolution via VCS
data.  First, each version of all source files in the project are
reconstituted from the differences stored within the VCS such that each
version of a file can be parsed by an appropriate language front end.  Each
front-end is configured to map the parsed code to an annotated term (or,
aterm)~\cite{brand00aterm} that represents a standardized serialization of the parse
tree or abstract syntax tree.  Mapping languages to a common aterm format allows
the downstream portions of our workflow to be language-agnostic to a large degree,
with minimal language-specific parameterization.  

Once we have code in an aterm format, we can then apply a structural
differencing algorithm between adjacent versions of each source file (e.g.,
version $n$ of file $f$ is compared to version $n+1$ of file $f$).  The result
of this is a forest of trees that represent the portions of the AST of file
$f$ that changed between versions at the structural level.  These changes can
either be code insertions, deletions, or mutations.  Our differencing is based
on the work of Yang~\cite{yang91diff} whose algorithm was designed for computing
differences between source code versions.  Yang's goal was to improve the
visual presentation of differences in textual diff tools, and our use of their
algorithm to provide input to further tree analysis algorithms is novel.

After reducing the sequence of differences stored in the VCS, we have a large
forest of trees each representing a change that occurred over the evolution of
the software.  At this point, we seek to relate each of these trees via a 
tree similarity metric.  This is achieved by using Yang's algorithm a second time,
but in this case we ignore the sequence of edit operations that it produces and
simply consume the quantitative similarity metric that it produces as a rough
estimate of how closely related two trees are.  (TODO: where to discuss the fact that
this is not a commutative operator).  A threshold parameter is defined in which
two trees with a similarity above the threshold are considered to be part of the same
cluster of difference tress.

Finally, once the set of differences are grouped into clusters of trees that
are similar up to the threshold, we perform antiunification on the entire
cluster to distill all members to a representative code pattern for the
cluster.  The antiunification algorithm as described by
Bulychev~\cite{bulychev08dupe} as part of the clonedigger project
was used, which itself is an implementation of the classical antiunitification
algorithm of Reynolds~\cite{reynolds69antiunification}.

In the following sections, we descibe the steps above in greater detail.

\subsection{Parsing and aterm generation}

One of the most challenging aspects of performing this kind of
study on arbitrary software packages is the availability of
robust language parsing tools.  In the absence of a common intermediate
representation or abstract syntax representation for popular languages, 
we adopted a standardized serialization format in the form of annotated
terms.  Generation of aterms was achieved via language-specific parsers.
In this work, we used the language-java and language-c parsers available
as open source libraries accessible via the Haskell programming language.

More on this stuff here.  Language to Aterm mapping details.

\subsection{Structural differencing}

Related work on structural change detection~\cite{chawathe96change}.  Tree
diff algorithms~\cite{tai79tree, selkow77tree}.  These may be very
related~\cite{weissgerber06identify, kim07automatic, neamtiu05understand} - I
need to quickly re-scan these to see the overlap.

\subsection{Tree similarity metric and clustering}

Given two trees $t_1$ and $t_2$, we would like to define a similarity metric
such that $d(t_1, t_2) \in [0,1]$, where a similarity of $1$ means that the
trees are identical, and $0$ represents maximal dissimilarity.  In Yang's
algorithm, a similarity score is provided for comparing $t_a$ and $t_b$. This
metric is order dependent, forcing the maximal score to be the size of the
first tree ($t_a$), even if $t_b$ is larger.  If the trees are identical, the
score will be exactly $|t_a|$, the number of nodes in $t_a$.  If they differ,
it will be strictly less than $|t_a|$.  As such, we define our distance
function to be $\frac{d(t_a, t_b)}{|t_a|}$.  This operator is not commutative,
since it is easy to find instances such that $\frac{d(t_b, t_a)}{|t_b|} \neq
\frac{d(t_a, t_b)}{|t_a|}$ when the trees are very different.

Once we have the set of changes that were detected from the VCS history,
we can generate a forest of trees $t_1, \cdots, t_n$ such that we can
compute the $n^2$ distances between all pairs to generate a distance matrix
$D$ where $D_{ij} = d(t_i, t_j)$.  Given a threshold value $\tau$, we can
produce a boolean matrix $D'$ where $D'_{ij} = d(t_i, t_j) > \tau$.  An
example matrix is shown in Figure~\ref{fig:boolmat} for changes observed
in the VCS for PROJECT X where $\tau = 0.9$.  

Given such a matrix, we would like to use information from the Yang
differencing algorithm to distinguish what kind of change each subtree
represents.  Three possibilities exist: a code insertion (a ``left-hole''),
code deletion (a ``right-hole''), and code mutation (a ``mismatch'').  Adding
this information to the difference matrix is important as it allows us to
further refine our view of the code evolution to distinguish code changes from
the insertion or removal of code that occurs over time. For example, when code
is being developed and grown, we expect to see a number of code insertions.
Similarly, when a mass refactoring occurs to simplify code, we would expect to
see a set of code deletions.  When a more subtle refinement occurs, such as
transposition of  code arguments or the addition of a conditional to refine
control flow, we would expect to see mismatches where the tree changes in a
way where the structural change can be thought of as a  coupled deletion and
insertion operation (such as changing a variable name in an expression).

\subsection{Anti-unification and template generation}

Once we have clusters of related code snippets in the form of related
subtrees, we can seek patterns that relate changes.  For example, say we have
a function call $foo()$ where each invocation of the function uses the same
parameters (e.g, $foo(x,y)$, where $x$ and $y$ are always the same). If we add
a new parameter at the end of each call where the variable passed in differs
each time (e.g., $foo(x,y,a)$ and $foo(x,y,b)$), we would like to abstract out
this change as $foo(x,y,VAR)$, where each instance of the change replaces
$VAR$ with whatever concrete variable is used at that point.  The anti-unification
algorithm is built for this purpose -- given two trees, it seeks the least-general
generalization of the pair and produces a triplet representing the generalized tree with
a metavariable introduced where the two differ, as well as a substitution set that allows
the metavariable to be replaced with the appropriate concrete subtree necessary to
reconstitute the two trees that were antiunified.

\section{Conclusions}

I conclude that this work kicked ass.

\section{Acknowledgments}

This work was supported in part by the US Department of
Energy Office of Science, Advanced Scientific Computing Research
contract no. XX-YYYYYYYYYY.  Additional support was provided by Galois,
Inc.

%\end{document}  % This is where a 'short' article might terminate

%ACKNOWLEDGMENTS are optional

%
% The following two commands are all you need in the
% initial runs of your .tex file to
% produce the bibliography for the citations in your paper.
\bibliographystyle{abbrv}
\bibliography{sigproc}  % sigproc.bib is the name of the Bibliography in this case
% You must have a proper ".bib" file
%  and remember to run:
% latex bibtex latex latex
% to resolve all references
%
% ACM needs 'a single self-contained file'!
%
%% %APPENDICES are optional
%% %\balancecolumns
%% \appendix
%% %Appendix A
%% \section{Headings in Appendices}
%% The rules about hierarchical headings discussed above for
%% the body of the article are different in the appendices.
%% In the \textbf{appendix} environment, the command
%% \textbf{section} is used to
%% indicate the start of each Appendix, with alphabetic order
%% designation (i.e. the first is A, the second B, etc.) and
%% a title (if you include one).  So, if you need
%% hierarchical structure
%% \textit{within} an Appendix, start with \textbf{subsection} as the
%% highest level. Here is an outline of the body of this
%% document in Appendix-appropriate form:
%% \subsection{Introduction}
%% \subsection{The Body of the Paper}
%% \subsubsection{Type Changes and  Special Characters}
%% \subsubsection{Math Equations}
%% \paragraph{Inline (In-text) Equations}
%% \paragraph{Display Equations}
%% \subsubsection{Citations}
%% \subsubsection{Tables}
%% \subsubsection{Figures}
%% \subsubsection{Theorem-like Constructs}
%% \subsubsection*{A Caveat for the \TeX\ Expert}
%% \subsection{Conclusions}
%% \subsection{Acknowledgments}
%% \subsection{Additional Authors}
%% This section is inserted by \LaTeX; you do not insert it.
%% You just add the names and information in the
%% \texttt{{\char'134}additionalauthors} command at the start
%% of the document.
%% \subsection{References}
%% Generated by bibtex from your ~.bib file.  Run latex,
%% then bibtex, then latex twice (to resolve references)
%% to create the ~.bbl file.  Insert that ~.bbl file into
%% the .tex source file and comment out
%% the command \texttt{{\char'134}thebibliography}.
%% % This next section command marks the start of
%% % Appendix B, and does not continue the present hierarchy
%% \section{More Help for the Hardy}
%% The acm\_proc\_article-sp document class file itself is chock-full of succinct
%% and helpful comments.  If you consider yourself a moderately
%% experienced to expert user of \LaTeX, you may find reading
%% it useful but please remember not to change it.
%% \balancecolumns
%% % That's all folks!
\end{document}
